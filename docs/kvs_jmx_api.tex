\subsection{JMX interfases }
KVS application registers MBeans called Kvs and Ring. MBean is object that simlular to JavaBean that represent resourse
that can be management using JMX techlology.
The Java Management Extensions (JMX) technology is a part of Java Platform that gives abillity to manage aplication remotely.
In order to connect to Kvs MBean you can use standart application as jconsole that provided with JDK and located in
 $JDK_HOME/bin, or others  that complies to the JMX specification.


\paragraph{KVS JMX interfase}
\begin{Methods}

\item [Read all feed as string]
allStr(fid:String):String

Return string representation of all entities in spesified feed.

\item [Export all data]
def save():String


Current version of KVS using RNG application as backend layer. As far as RNG is distrebuted data store it's not possible
to backup or migrate data form server to server only with copy-past dirrectory with persisted data. Even more any copy
of RNG's data will has no any sense becase data in RNG storage is partitioned and particular node can be asquered only for
defined range of keys in store, and in case

RNG become readonly to keep consistency.

\item [Load file]
def load(path:String):Any

During loading alse readonly. Data from loaded file has higher priority compare to already stored. KVS become readable
and writeable when loading is finished.
Quorum should be satisfied otherwise data iterated and ignored on write opperation. This condition can be checked by comparing
 W property from quorum configuration and currently reachable nodes in cluster.
 Important note that loading of file that was created on differ version of services can lead to broken data. It's caused
  because RNG persist data in bytes so all entities that serilisated before saving goes throught serialisation-deserialisation.
  That's why KVS not supported compatability between differ version of schema.

\end


\paragraph{RNG JMX interfase}
\begin{Methods}
\item
def get(key:String): String
\item
  def put(key:String, data: String):String
\item
def delete(key:String):Unit
\end

