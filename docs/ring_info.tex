\section{Ring datastore}

Scala implementation of Kai (originally implemented in erlang).
Kai is a distributed key-value datastore, which is mainly inspired
by Amazon's Dynamo. Ring is implemented on top of akka and injected as akka extension.

\section*{Overview}

To reach fault tolerance and scalability ring resolve next problems:

\begin{description}

\item[Problems:] Technique
\item[Membership and failure detection: ] reused akka's membership events that uses gossip for communication. FD also reused from akka.
\item[Data partitioning:] consistent hashing.
\item[High availability to wright:] vector clocks increase number of write opperation to merge data on read opperation.
\item[Handling nodes failures:] gossip protocol.

\end{description}

\section*{Consistent hashing}

To figure out where the data for a particular key goes in that cluster you need to apply a hash function to the key.
Just like a hashtable, a unique key maps to a value and of course the same key will always return the same hash code.
In very first and simple version of this algorithm the node for particular key is determined by hash(key) mod n, where n is a number of
nodes in cluster. This works well and trivial in implementation but when new node join or removed from cluster we got a problem, every object is hashed to a new location.
The idea of the consistent hashing algorithm is to hash both node and key using the same hash function.
As result we can map the node to an interval, which will contain a number of key hashes. If the node is removed
then its interval is taken over by a node with an adjacent interval.

\section*{Vector clocks}

Vector clocks is an algorithm for generating a partial ordering of events in a distributed system and detecting causality violations. (from wikipedia.org)
Vector clocks help us to determine order in which data writes was occurred. This provide ability to write data from one node and after that 
merge version of data. Vector clock is a list of pairs of node name and number of changes from this node.
When data writs first time the vector clock will have one entity ( node-A : 1). Each time data amended the counter is incremented.

\section*{Quorum}

Quorum determines the number of nodes that should be participated in operation. Quorum-like system configured by values: R ,W and N. R is the
minimum number of nodes that must participate in a successful read operation. W is the minimum number of nodes that must participate.
N is a preference list, the max number of nodes that can be participated in operation. Also quorum can configure balance of latency
for read and write operation.
 In order to keep data strongly consistent configuration should obey rules:
 \begin{lstlisting}[language=bash]
  1) R + W > N
  2) W > V/2
 \end{lstlisting}



