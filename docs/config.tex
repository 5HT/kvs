\section{Ring Datastore}
\paragraph{}
Ring Application is available as akka extension and makes your system part of the highly available, fault tolerant data distrubution cluster.

\paragraph{}
To configure \texttt{RNG} application on your cluster the next config options are available:

\paragraph{}
Default configuration provide single mode configuration, other default properties listed below:

\begin{json}
ring{
  quorum=[1,1,1]  #N,W,R.
  buckets=1024
  virtual-nodes=128
  hashLength=32
  gather-timeout = 3
  ring-node-name="ring_node"
  leveldb {
    native = true
    dir = "fav-data"
    checksum = false
    fsync = false
  }
}
\end{json}

\paragraph{}
In case default value is suitable for particular deployment, rewrite is not needed.

\begin{description}
\item[quorum] Configured by template N,W,R where

\begin{description}
\item[N] Number of nodes in bucket(in other words the number of copies).
\item[R] Number of nodes that must  be participated in successful read operation.
\item[W] Number of nodes for successful write.
\end{description}

To keep data consistent the quorums have to obey the following rules:
\begin{enumerate}
\item R + W > N
\item W > N/2
\end{enumerate}

Or use the next hint:
\begin{itemize}
\item single node cluster [1,1,1]
\item two nodes cluster [2,2,1]
\item 3 and more nodes cluster [3,2,2]
\end{itemize}

if quorum fails on write operation, data will not be saved.
So in case if 2 nodes and [2,2,1] after 1 node down the cluster becomes not writeable and readable.
\item[buckets] Number of buckets for key. Think about this as about size of HashMap. In current implementation this value should not 
 be changed after first setup.
\item[virtual\-nodes] Number of virtual nodes for each physical. In current implementation this value should not
 be changed after first setup.
\item[hashLength] Lengths of hash from key. In current implementation this value should not
 be changed after first setup.
\item[gather\-timeout] Number of seconds that requested cluster will wait for response from another nodes.
\item[ring\-node\-name] Role name that mark node as part of ring.
\item[leveldb] Configuration of levelDB database used as backend for ring.
\end{description}


\section{LevelDB}
Where are options to configure the backend under ring.leveldb configuration section:
\begin{description}
\item[native] usage of native or java implementation if LeveDB
\item[dir] directory location for levelDB storage.
\item[checksum] checksum verification of all data that is read from the file system.
\item[fsync] if true levelDB will synchronise data to disk immediately.
\end{description}